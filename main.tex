% $Id: template.tex 11 2007-04-03 22:25:53Z jpeltier $

\documentclass{vgtc}                          % final (conference style)
% \documentclass[review]{vgtc}                 % review
%\documentclass[widereview]{vgtc}             % wide-spaced review
%\documentclass[preprint]{vgtc}               % preprint
%\documentclass[electronic]{vgtc}             % electronic version

%% Uncomment one of the lines above depending on where your paper is
%% in the conference process. ``review'' and ``widereview'' are for review
%% submission, ``preprint'' is for pre-publication, and the final version
%% doesn't use a specific qualifier. Further, ``electronic'' includes
%% hyperreferences for more convenient online viewing.

%% Please use one of the ``review'' options in combination with the
%% assigned online id (see below) ONLY if your paper uses a double blind
%% review process. Some conferences, like IEEE Vis and InfoVis, have NOT
%% in the past.

%% Figures should be in CMYK or Grey scale format, otherwise, colour
%% shifting may occur during the printing process.

%% it is recomended to use ``\cref{sec:bla}'' instead of ``Fig.~\ref{sec:bla}''
\graphicspath{{figures/}{pictures/}{images/}{./}} % where to search for the images

\usepackage{times}                     % we use Times as the main font
\renewcommand*\ttdefault{txtt}         % a nicer typewriter font

%% Only used in the template examples. You can remove these lines.
\usepackage{tabu}                      % only used for the table example
\usepackage{booktabs}                  % only used for the table example
\usepackage{lipsum}                    % used to generate placeholder text
\usepackage{mwe}                       % used to generate placeholder figures

%% We encourage the use of mathptmx for consistent usage of times font
%% throughout the proceedings. However, if you encounter conflicts
%% with other math-related packages, you may want to disable it.
\usepackage{mathptmx}                  % use matching math font

%% If you are submitting a paper to a conference for review with a double
%% blind reviewing process, please replace the value ``0'' below with your
%% OnlineID. Otherwise, you may safely leave it at ``0''.
\onlineid{0}

%% declare the category of your paper, only shown in review mode
\vgtccategory{Research}

%% allow for this line if you want the electronic option to work properly
\vgtcinsertpkg

%% In preprint mode you may define your own headline. If not, the default IEEE copyright message will appear in preprint mode.
%\preprinttext{To appear in an IEEE VGTC sponsored conference.}

%% This adds a link to the version of the paper on IEEEXplore
%% Uncomment this line when you produce a preprint version of the article
%% after the article receives a DOI for the paper from IEEE
%\ieeedoi{xx.xxxx/TVCG.201x.xxxxxxx}


%% Paper title.

\title{Accessible Visualization Design Using Matplotlib }

%% This is how authors are specified in the conference style

%% Author and Affiliation (single author).
%%\author{Roy G. Biv\thanks{e-mail: roy.g.biv@aol.com}}
%%\affiliation{\scriptsize Allied Widgets Research}

%% Author and Affiliation (multiple authors with single affiliations).
%%\author{Roy G. Biv\thanks{e-mail: roy.g.biv@aol.com} %
%%\and Ed Grimley\thanks{e-mail:ed.grimley@aol.com} %
%%\and Martha Stewart\thanks{e-mail:martha.stewart@marthastewart.com}}
%%\affiliation{\scriptsize Martha Stewart Enterprises \\ Microsoft Research}

%% Author and Affiliation (multiple authors with multiple affiliations)

\author{Hannah Aizenman \thanks{email:haizenman@gradcenter.cuny.edu}\\%
        \scriptsize Computer Science, The Graduate Center, CUNY
\and Kyle Sunden \thanks{e-mail: contact@ksunden.space}\\%
        \scriptsize Matplotlib %
\and Elliot Sales de Andrade  \thanks{e-mail: elliot quantum.analyst@gmail.com }\\%
        \scriptsize Matplotlib %
% NOTE: would maybe make travel funding easier
%\and Mikael Vejdemo-Johansson \thanks{e-mail: mvj@math.csi.cuny.edu}\\%
%        \scriptsize  Mathematics, CUNY College of Staten Island %
\and Thomas Caswell \thanks{e-mail: tcaswell@bnl.gov}
        \parbox{1.8in}{\scriptsize\centering National Synchrotron Light Source II \\ Brookhaven National Lab}
}

%% A teaser figure can be included as follows
%\teaser{
%  \centering
%  \includegraphics[width=\linewidth]{CypressView}
%  \caption{In the Clouds: Vancouver from Cypress Mountain.}
%  \label{fig:teaser}
%}

%% Abstract section.
\abstract{
    Visualization libraries attempt to provide the tools to make accessible
    visualzations. What does that mean?

    Key points:

    color/texture
    Components can build accessible visualizations.
    colorblind safe colormaps and color sequences,
    customizable dash and hatch patterns,
    customizable markers
    text:
    customizable fonts
    internationalization (libraqm)

    alt-text is complicated
    multiple backends, interactive backends are

    where architecture can help:
    Matplotlib's artist model means there's a core object to query for information
    to build automatic description. question is what info is needed

} % end of abstract

%% Keywords that describe your work. Will show as 'Index Terms' in journal
%% please capitalize first letter and insert punctuation after last keyword.
\keywords{accessiblity, visualization library design, internationalization}

%% Copyright space is enabled by default as required by guidelines.
%% It is disabled by the 'review' option or via the following command:
% \nocopyrightspace

%%%%%%%%%%%%%%%%%%%%%%%%%%%%%%%%%%%%%%%%%%%%%%%%%%%%%%%%%%%%%%%%
%%%%%%%%%%%%%%%%%%%%%% START OF THE PAPER %%%%%%%%%%%%%%%%%%%%%%
%%%%%%%%%%%%%%%%%%%%%%%%%%%%%%%%%%%%%%%%%%%%%%%%%%%%%%%%%%%%%%%%%

\begin{document}

%% The ``\maketitle'' command must be the first command after the
%% ``\begin{document}'' command. It prepares and prints the title block.

%% the only exception to this rule is the \firstsection command
\firstsection{Introduction}

\maketitle

%% \section{Introduction} %for journal use above \firstsection{..} instead

Matplotlib is a building block \cite{wongsuphasawatNavigatingWideWorld2021}
Python library that developers can use to build static, animated, and interactive
visualizations. Matplotlib out of the box supports making accessible design choices
because it is designed to be general purpose while the support for static and dynamic visualizations
complicates supporting alt text in a consistent manner.

%note: insert transition sentence about MPL architecture facilitating accesibility

Every visual element in an image produced by Matplotlib -  every line, text, image - is
backed by an object called an Artist \cite{hunterArchitectureOpenSource}. Each artist knows
the properties of the element it is abstracting and if it is the child of another artist, such
as if it is a line inside a figure. This information could be exposed to accessibility tools,
such as alt-text generators or image navigators - through an API, which the Matplotlib
developers would welcome feedback on.


\section{Visual Variables: Color \& Texture}

Customize all the things - color and texture

% maybe add petroff figure here
\cite{petroffAccessibleColorSequences2024}, colorblind safe and dark mode colormaps
\cite{leeMpl_pe_pattern_monster011Documentation}

\section{Text}
No really customize all the things - including text!
%% Add libraqm figure

\subsection{Formatting}
\cite{elavskyBokehAccessibilityAudit} complains about lack of autoformatting, which MPL has in
spades, also zoom and rasterization

\section{Alt text \& Aria}
%https://bokeh-a11y-audit.readthedocs.io/#help-developers-succeed -
defaults/guardrails/validating alt text

\subsection{Static Backends}
Standards what standards?
\subsection{Interactive Backends}
Rendered as PNG - infeasible to implement GUI library in mpl, positive is folks can
build whatever interactions they want/multiple ways of accessing the data (\cite{BokehAccessibilityAudit})


\subsection{API for accesibility tools}
Exploit Artist model \cite{hunterMatplotlib2DGraphics2007,hunterArchitectureOpenSource}
%cite{data prototype} provides system state https://bokeh-a11y-audit.readthedocs.io/#c-interactive-capabilities-and-system-state-are-not-clear

%https://bokeh-a11y-audit.readthedocs.io/#c-interactive-capabilities-and-system-state-are-not-clear

\section{Conclusion}


%% if specified like this the section will be committed in review mode
\acknowledgments{
The authors wish to thank A, B, and C. This work was supported in part by
a grant from XYZ.}

%\bibliographystyle{abbrv}
%\bibliographystyle{abbrv-doi}
%\bibliographystyle{abbrv-doi-narrow}
%\bibliographystyle{abbrv-doi-hyperref}
%\bibliographystyle{abbrv-doi-hyperref-narrow}
\bibliographystyle{plain}
\bibliography{references}
\end{document}
