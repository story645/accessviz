% $Id: template.tex 11 2007-04-03 22:25:53Z jpeltier $

\documentclass{vgtc}                          % final (conference style)
% \documentclass[review]{vgtc}                 % review
%\documentclass[widereview]{vgtc}             % wide-spaced review
%\documentclass[preprint]{vgtc}               % preprint
%\documentclass[electronic]{vgtc}             % electronic version

%% Uncomment one of the lines above depending on where your paper is
%% in the conference process. ``review'' and ``widereview'' are for review
%% submission, ``preprint'' is for pre-publication, and the final version
%% doesn't use a specific qualifier. Further, ``electronic'' includes
%% hyperreferences for more convenient online viewing.

%% Please use one of the ``review'' options in combination with the
%% assigned online id (see below) ONLY if your paper uses a double blind
%% review process. Some conferences, like IEEE Vis and InfoVis, have NOT
%% in the past.

%% Figures should be in CMYK or Grey scale format, otherwise, colour
%% shifting may occur during the printing process.

%% it is recomended to use ``\cref{sec:bla}'' instead of ``Fig.~\ref{sec:bla}''
\graphicspath{{figures/}{pictures/}{images/}{./}} % where to search for the images

\usepackage{times}                     % we use Times as the main font
\renewcommand*\ttdefault{txtt}         % a nicer typewriter font

%% Only used in the template examples. You can remove these lines.
\usepackage{tabu}                      % only used for the table example
\usepackage{booktabs}                  % only used for the table example
\usepackage{lipsum}                    % used to generate placeholder text
\usepackage{mwe}                       % used to generate placeholder figures

%% We encourage the use of mathptmx for consistent usage of times font
%% throughout the proceedings. However, if you encounter conflicts
%% with other math-related packages, you may want to disable it.
\usepackage{mathptmx}                  % use matching math font

%% things that need to be addressed
\newcommand{\note}[1]{\textcolor{magenta}{#1}}

%% If you are submitting a paper to a conference for review with a double
%% blind reviewing process, please replace the value ``0'' below with your
%% OnlineID. Otherwise, you may safely leave it at ``0''.
\onlineid{0}

%% declare the category of your paper, only shown in review mode
\vgtccategory{Research}

%% allow for this line if you want the electronic option to work properly
\vgtcinsertpkg

%% In preprint mode you may define your own headline. If not, the default IEEE copyright message will appear in preprint mode.
%\preprinttext{To appear in an IEEE VGTC sponsored conference.}

%% This adds a link to the version of the paper on IEEEXplore
%% Uncomment this line when you produce a preprint version of the article
%% after the article receives a DOI for the paper from IEEE
%\ieeedoi{xx.xxxx/TVCG.201x.xxxxxxx}


%% Paper title.

\title{Accessible Visualization Design using Matplotlib}

%% This is how authors are specified in the conference style

%% Author and Affiliation (single author).
%%\author{Roy G. Biv\thanks{e-mail: roy.g.biv@aol.com}}
%%\affiliation{\scriptsize Allied Widgets Research}

%% Author and Affiliation (multiple authors with single affiliations).
%%\author{Roy G. Biv\thanks{e-mail: roy.g.biv@aol.com} %
%%\and Ed Grimley\thanks{e-mail:ed.grimley@aol.com} %
%%\and Martha Stewart\thanks{e-mail:martha.stewart@marthastewart.com}}
%%\affiliation{\scriptsize Martha Stewart Enterprises \\ Microsoft Research}

%% Author and Affiliation (multiple authors with multiple affiliations)

\author{Hannah Aizenman \thanks{email:haizenman@gradcenter.cuny.edu}\\%
        \scriptsize Computer Science, The Graduate Center, CUNY
\and Kyle Sunden \thanks{e-mail: contact@ksunden.space}\\%
        \scriptsize Matplotlib %
\and Elliot Sales de Andrade  \thanks{e-mail: elliot quantum.analyst@gmail.com }\\%
        \scriptsize Matplotlib %
% NOTE: would maybe make travel funding easier
%\and Mikael Vejdemo-Johansson \thanks{e-mail: mvj@math.csi.cuny.edu}\\%
%        \scriptsize  Mathematics, CUNY College of Staten Island %
\and Thomas Caswell \thanks{e-mail: tcaswell@bnl.gov}
        \parbox{1.8in}{\scriptsize\centering National Synchrotron Light Source II \\ Brookhaven National Lab}
}

%% A teaser figure can be included as follows
%\teaser{
%  \centering
%  \includegraphics[width=\linewidth]{CypressView}
%  \caption{In the Clouds: Vancouver from Cypress Mountain.}
%  \label{fig:teaser}
%}

%% Abstract section.
\abstract{
    Visualization libraries attempt to provide the tools to make accessible
    visualzations. What does that mean?

    Key points:

    color/texture
    Components can build accessible visualizations.
    colorblind safe colormaps and color sequences,
    customizable dash and hatch patterns,
    customizable markers
    text:
    customizable fonts
    internationalization (libraqm)

    alt-text is complicated
    multiple backends, interactive backends are

    where architecture can help:
    Matplotlib's artist model means there's a core object to query for information
    to build automatic description. question is what info is needed

} % end of abstract

%% Keywords that describe your work. Will show as 'Index Terms' in journal
%% please capitalize first letter and insert punctuation after last keyword.
\keywords{accessiblity, visualization library design, internationalization}

%% Copyright space is enabled by default as required by guidelines.
%% It is disabled by the 'review' option or via the following command:
% \nocopyrightspace

%%%%%%%%%%%%%%%%%%%%%%%%%%%%%%%%%%%%%%%%%%%%%%%%%%%%%%%%%%%%%%%%
%%%%%%%%%%%%%%%%%%%%%% START OF THE PAPER %%%%%%%%%%%%%%%%%%%%%%
%%%%%%%%%%%%%%%%%%%%%%%%%%%%%%%%%%%%%%%%%%%%%%%%%%%%%%%%%%%%%%%%%

\begin{document}

%% The ``\maketitle'' command must be the first command after the
%% ``\begin{document}'' command. It prepares and prints the title block.

%% the only exception to this rule is the \firstsection command
\firstsection{Introduction}

\maketitle

%% \section{Introduction} %for journal use above \firstsection{..} instead

Matplotlib is a building block \cite{wongsuphasawatNavigatingWideWorld2021}
Python library that developers can use to build static, animated, and interactive
visualizations. Matplotlib supports making accessible design choices
because it is designed to be general purpose while the support for static and dynamic visualizations complicates supporting alt text in a consistent manner. Every visual element in an image produced by Matplotlib -  every line, text, image - is backed by an object called an Artist \cite{hunterArchitectureOpenSource}. Each artist knows the properties of the element it is abstracting and if it is the child of another artist, such as if it is a line inside a figure. This information could be exposed to accessibility tools, such as alt-text generators or image navigators - through an API and the Matplotlib developers would welcome feedback on what information would be useful to expose.



\section{Visual Variables: Color and Texture}

Matplotlib allows users to customize aesthetics to fit many accessibility needs and
sets accessible default colormaps, sequences, and texture patterns to aide users who do not have expertise in building accessible visualizations.
\subsection{Color}
\begin{figure}
        \centering
        \includegraphics[width=\linewidth, alt={visual listing of the colors in the tableau and Petroff named color sequences}]{color_sequences.png
        }
        \caption{Each set of circles displays the colors in the respective color sequence. Accessible color sequences such as tableau and Petroff are designed such that each color is perceptually distinguishable from the other colors in the sequence.}
        \label{fig:color_cycle}
\end{figure}

Matplotlib's color sequences (palettes) are used by the plotting methods to automatically assign colors to visual elements. When plotting multiple line plots, for example, the first line takes the first color of the sequence, the second line takes the second color, the nth line the nth color, and the sequence cycles once the colors are exhausted. As shown in \autoref{fig:color_cycle}, Matplotlib's default color sequence is the tableau color sequence, which is designed to be accessible to people with color vision deficiencies (CVD) \cite{stoneHowWeDesigned}. The Petroff color sequences \cite{petroffAccessibleColorSequences2024} were added in versions 3.10.0 and 3.11.0 and are designed with the aide of machine learning to balance accessibility constraints with aesthetic considerations.


\begin{figure}
        \centering
        \includegraphics[width=\linewidth, alt={The berlin colormap goes dark blue to reddish pink, the managua colormap goes sky blue to orange, and the
        vanimo colormap goes bright green to bright pink. All three colormaps have a black center.}]{darkcmap.png}
        \caption{The berling, managua, and venimo are dark-mode diverging colormaps,
        with minimum lightness at the center, and maximum lightness at the extremes.}
        \label{fig:dark_color_map}
\end{figure}

All of Matplotlib's sequential and diverging colormaps are perceptually uniform and the default colormap `Viridis' is designed to be distinguishable for people with most forms of CVD \cite{enthoughtBetterDefaultColormap2015}. Additionally, the diverging dark mode colormaps \cite{crameriScientificColourMaps2023} in \autoref{fig:dark_color_map} were added in version 3.10.0 to support dark mode displays.

If these color sequences or maps do not suit the situation, Matplotlib also allows users to customize the color of every visual element in an image through the Artist interface. The colors can be specified in RGB/RGBA, HEX, grayscale, x11/CSS4, and a few library specific formats\cite{SpecifyingColorsMatplotlib}. The colors module supports creating custom linear and categorical colormaps, the normalization module provides a variety of methods for encoding numbers as colors, and the new colorizer module allows for even more customizability.


\subsection{Texture}

\begin{figure}[!h]
        \centering
        \includegraphics[width=\linewidth, alt={}]{sphx_glr_hatch_style_reference_003.png}
        \caption{New hatch patterns can be created by composing existing patterns and any hatch pattern can be used as a fill in any matplotlib \texttt{Patch} object (broadly most 2D shapes generated by the library).}
        \label{fig:hatch_patterns}
\end{figure}

In addition to or in place of color, Matplotlib has rich support for using textures to add visual distinguishability in a graphic Matplotlib supports almost 40
named marker styles, markers created from Tex symbols, and custom markers created from
paths (arbitrary shapes) \cite{MarkerReferenceMatplotlib}. Additionally, Matplotlib
provides 4 named line styles ('solid', 'dotted', 'dashed', 'dashdot') and allows users
to configure the on/off sequence to create arbitrary patterns. Matplotlib also supports
10 hatch patterns (\texttt{/|-+*.xoO}) that, as shown in \autoref{fig:hatch_patterns}, can be combined and can also be made more dense by repeating the specification,
i.e. 'oo' yields more closely spaced circles. Additionally, Matplotlib supports generating custom fill patterns, as exemplified by third party libraries such as
\cite{leeMpl_pe_pattern_monster011Documentation}.

\section{Text}
As with color and texture, Matplotlib's fonts allow a very high level of customization so that users can choose the fonts, styles, weights, sizes, and colors best suited for their audience\cite{FontsMatplotlibMatplotlib}. In addition, Matplotlib also has a robust API for formatting ticks \cite{MatplotlibtickerMatplotlib3103}. The default font \texttt{DejaVu sans}\cite{DejaVuFonts} provides characters that are sans serif, unembellished, and distinguishable but does not guarantee consistent stroke width or symbol spacing. There is ongoing work, aimed for version 3.11.0, to improve the font system \cite{https://github.com/orgs/matplotlib/projects/7}.

\begin{figure}[!h]
        \centering
        \includegraphics[width=\linewidth, alt={}]{libraqm.png}
        \caption{}
        \label{fig:libraqm}
\end{figure}

One major goal of the 3.11 font work is to better support non-Latin alphabets because graphs can be more easily created and understood when they incorporate conventions from the user's socio-cultural experiences \cite{byrneAcquiredCodesMeaning2016}. This support is being added through the adoption of \texttt{libraqm} \cite{HOSTOmanLibraqm2025} as the text engine. As shown in \autoref{fig:libraqm}, \texttt{libraqm} supports right-to-left and left-to-right texts, and  parsing and placing letters, vowels, diacritics, and variant character forms correctly\cite{WhatHarfBuzzHarfBuzz,WhyNeedShaping}; therefore migrating to \texttt{libraqm} will allow Matplotlib to more easily support most of the writing systems covered by the Unicode standard.

\section{Alt text \& Aria}
%https://bokeh-a11y-audit.readthedocs.io/#help-developers-succeed -
defaults/guardrails/validating alt text

\subsection{Static Backends}
Standards what standards?
\subsection{Interactive Backends}
Rendered as PNG - infeasible to implement GUI library in mpl, positive is folks can
build whatever interactions they want/multiple ways of accessing the data (\cite{BokehAccessibilityAudit})


\subsection{API for accesibility tools}
Exploit Artist model \cite{hunterMatplotlib2DGraphics2007,hunterArchitectureOpenSource}
%cite{data prototype} provides system state https://bokeh-a11y-audit.readthedocs.io/#c-interactive-capabilities-and-system-state-are-not-clear

%https://bokeh-a11y-audit.readthedocs.io/#c-interactive-capabilities-and-system-state-are-not-clear

\section{Conclusion}
Matplotlib provides accessible defaults and options for colors, a readable default font, and many ways to customize the graphic to make the various elements more accessible. There are architectural challenges that make it difficult to make interactive visualizations accessible, but but the artist tree stores information that could potentially be exposed to accessibility tools to compensate for the inability to introspect GUI elements.

\section{Figure Credits}
\label{sec:figure_credits_inst}

\autoref{fig:color_cycle} is a modification of the Matplotlib named color sequences example\cite{NamedColorSequences},
\autoref{fig:dark_color_map} is a modification of the example in the 3.10 release notes \cite{WhatsNewMatplotlib}, \autoref{fig:hatch_patterns} is one of the examples in the Matplotlib hatch reference examples \cite{HatchStyleReference}.


%% if specified like this the section will be committed in review mode
\acknowledgments{
The authors wish to thank the numerous people who have contributed this work to Matplotlib. Hannah is thankful for the guidance the Bokeh accessibility audit\cite{elavskyBokehAccessibilityAudit} provided on how visualization libraries can help users build accessible visualizations. The font and alt text in metadata work is supported in part by a NumFocus Small Development Grant.}

%\bibliographystyle{abbrv}
%\bibliographystyle{abbrv-doi}
%\bibliographystyle{abbrv-doi-narrow}
%\bibliographystyle{abbrv-doi-hyperref}
%\bibliographystyle{abbrv-doi-hyperref-narrow}
\bibliographystyle{plain}
\bibliography{references}
\end{document}
